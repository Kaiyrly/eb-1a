% \UnnumberedTitle{Statement from Mr. Taximov detailing plans on how he intends to continue work in the United States}

% My name is Askar Taximov. I am the beneficiary of this I-140 Immigrant Petition for Alien Worker, seeking EB-1A classification as an individual of extraordinary ability. I intend to continue my professional work in the United States in the area of urban digital infrastructure and IT sector, focusing on practical Smart City solutions for local governments and businesses.

% I will concentrate on U.S. municipalities for affordable yet effective IT systems. My goal is to establish both consulting and product activities centered on deploying city-incident management platforms based on the \textbf{NextITSM} architecture and my implementation experience with \textbf{iKOMEK 2.0}.

% I plan to collaborate directly with U.S. municipalities to adapt and implement the Smart City solutions my team and I have developed. My initial focus will be the rollout of a \textbf{cloud platform} for resident appeals intake and processing, \textbf{integration of video analytics}, \textbf{AI-assisted analysis of appeals} and \textbf{operational monitoring dashboards} for city departments. These solutions are particularly relevant for jurisdictions up to roughly 500,000 residents, where budgets are constrained but the need for digitalization is acute.

% In parallel, I will launch a \textbf{pilot project} with one U.S. city to adapt and test a \textbf{Smart Response} module (by analogy with iKOMEK). The platform will follow \textbf{low-code principles}, use an \textbf{open-source stack} and offer a \textbf{flexible licensing model} to ensure sustainability and attractiveness for local administrations.

% Through these technologies and methods, I aim to deliver concrete benefits on problems common to small and mid-sized cities: overloaded municipal services, the absence of a unified digital environment for receiving and analyzing resident appeals, fragmented data and outdated channels between residents and administration. The solutions I have already validated can be \textbf{adapted to U.S. conditions} to close this technology gap—accelerating digitalization of urban infrastructure, improving management efficiency and reducing budgetary costs—thereby \textbf{benefiting the U.S.}.

% My U.S. work will also include \textbf{knowledge exchange}: publishing implementation methods and effective models of digital transformation. In the longer term, I plan to pursue a \textbf{doctoral program} at a U.S. technical university in \textbf{Urban Digital Infrastructure} to deepen applied research. My objective is not only to build a product, but to contribute to a \textbf{sustainable digital ecosystem} in which technology serves people and cities become genuinely smart and safe.

% This plan is a direct continuation of my established professional path and demonstrates how my work in the United States will contribute to urban IT infrastructure, digital governance and economic efficiency by improving municipal systems and the safety of the urban environment.

% \begin{flushright}
% \textbf{Sincerely,}\\
% Askar Boranbayevich Taximov
% \end{flushright}

\UnnumberedTitle{Statement from Mr. Taximov detailing plans on how he intends to continue work in the United States}

My name is Askar Taximov. I am the beneficiary of this I-140 Immigrant Petition for Alien Worker, seeking EB-1A classification as an individual of extraordinary ability. I intend to continue my professional work in the United States in IT project management in the field of urban digital infrastructure, innovation and delivery for local governments.

I will build a consulting-and-product practice dedicated to municipal IT project management and execution. The practice will deliver incident and service-management platforms modeled on my prior NextITSM architecture and the operating model developed and proven at iKOMEK 2.0. Each engagement will apply disciplined delivery mechanics requirements baselining and scope control; governance charters; KPI/SLA taxonomies; risk, issue and change management; iterative release trains; data-quality and machine learning operations (MLOps) pipelines; and security-by-design to ensure schedule, budget and benefit realization in production environments.

I plan to collaborate directly with U.S. municipalities to adapt, implement and manage Smart City components that I have validated in practice, using modern IT project-management methods and tooling Agile delivery, KPI and service-level agreement (SLA) governance, continuous integration and continuous delivery (CI/CD), and machine learning operations (MLOps) to achieve goals efficiently and expeditiously. Initial deployments will prioritize: (i) a cloud platform for omnichannel citizen-appeals intake, routing and service-level agreement (SLA) tracking; (ii) artificial intelligence (AI)-assisted triage and analytics for pattern detection and demand forecasting; (iii) video analytics for public-safety and operations use cases; and (iv) executive dashboards for transparent performance management. These solutions are particularly relevant for jurisdictions up to approximately 1 million residents, where budgets are constrained but the need for digitalization is acute.

In parallel, I will launch a pilot “Smart Response” program with one U.S. city, adapting the iKOMEK approach to local constraints. The platform will follow low-code principles, use an open-source-first modular stack, and offer a flexible licensing model to keep total cost of ownership low and long-term operations sustainable for local administrations. This work is designed to deliver tangible outcomes for problems common to small and mid-sized cities, including overloaded municipal services, the lack of a unified digital environment for receiving and resolving resident appeals, fragmented data, and outdated channels between residents and administration. The same methods and architectures that proved effective in my prior programs can be adapted to U.S. conditions to close this technology gap accelerating digitalization of city infrastructure, improving management efficiency and lowering cost-to-serve thereby benefiting the United States.

The implementation of my projects will directly contribute to job creation and the strengthening of economic activity. In the course of this work, I plan to hire professionals in key areas, including project managers, analysts, data engineers, implementation specialists and technical support staff. This will generate formal employment, increase tax revenues and provide a direct economic benefit. Part of these teams will be established in mid-sized cities using a hybrid work model that combines on-site and remote participation. This approach enables the creation of jobs and economic activity not only in major metropolitan areas but also in cities with population up to one million residents, ensuring a stable and positive impact on local economies. 

This plan represents a direct continuation of my established professional path in IT infrastructure development, digital governance and economic efficiency. My work in the United States will contribute by improving municipal systems, strengthening cybersecurity practices, creating jobs in underserved regions and notably enhancing the quality and responsiveness of city services. With my expertise as an IT project manager, I am well positioned to execute this plan effectively and achieve its stated goals.


\begin{flushright}
\textbf{Sincerely,}\

Askar  Taximov
\end{flushright}

\pagebreak

% \section{\mrls{} proposed employment has both substantial merit and national importance for the United States}

% \label{sec:Benefit}







% The labor market demands skilled IT Project Managers. The U.S.\ Bureau of Labor Statistics reports a \underline{2024 median pay of \$100{,}750} for this role (BLS, 2024). Employment will grow \underline{6\%} from \underline{2024 to 2034} (BLS, 2024). This rate exceeds the average for all jobs (BLS, 2024). High wages prove that employers value this talent (BLS, 2024). Companies need these experts to build digital systems (BLS, 2024). Implication for job creation: executing programs of this kind predictably requires \underline{local hiring} across complementary roles—developers, delivery leads, site reliability engineering (SRE) / development and operations (DevOps) specialists, data engineers, information technology service management (ITSM) administrators, cybersecurity analysts—with ongoing U.S.-based operations rather than one-off deployments (BLS, 2024). This approach creates skilled jobs in digitalization and modernization (BLS, 2024).
% Companies are adopting tools quickly, and the market will reach \underline{\$20.47\,billion by 2030} (Grand View Research, 2023). This represents an annual growth rate of \underline{15.7\%} (Grand View Research, 2023). \underline{North America} leads the sector. The region held \underline{41.97\%} of the market in 2022 (Grand View Research, 2023).
% Research measures the cost of weak performance. Organizations waste \underline{\$114 million} for every \underline{\$1 billion} they spend (PMI, 2020). Later reports estimate that the world loses \underline{\$2 trillion} each year (PMI, 2024). This equals about \$1 million every 20 seconds (PMI, 2024).
% Large risk studies show extreme outcomes in IT portfolios (Flyvbjerg \& Budzier, 2022). Researchers observe "black swan" events and systematic overruns. These failures affect every sector, including the public sector. These patterns persist across multiple dimensions of project delivery (Flyvbjerg et al., 2025).
% Work methods are changing. Gartner predicts that AI will do \underline{80\%} of routine tasks by \underline{2030} (HBR, 2023). Humans will focus on strategy and value instead (HBR, 2023). These facts highlight the need for disciplined IT management. The United States needs experts in AI, analytics, and cybersecurity. This field supports the national economy.
% These sources confirm the value of disciplined IT management. The work combines AI, analytics, and cybersecurity. This field is critical to the U.S.\ economy.



% Cross-domain assessments indicate that U.S. municipalities face escalating pressures stemming from aging infrastructure and climate-related shocks (SmartBrief, 2025). These challenges are compounded by demographic expansion and labor-force constraints that limit modernization capacity (Zacua Ventures, 2024). These converging factors underscore the growing need for resilient and digitally enabled service-delivery systems (Lincoln Institute of Land Policy, 2024).
% Independent evaluations of urban-technology adoption indicate that Smart City interventions can raise operational efficiency on the order of \underline{20-30\%} (McKinsey, 2018). These interventions enable large fiscal savings through optimized assets and processes (McKinsey, 2018).
% Correspondingly, the U.S. Smart Cities market is projected to reach approximately \underline{\$260.9\,billion by 2028} (MarketsandMarkets, 2024). This projection evidences strong and growing domestic demand for interoperable, secure platforms (MarketsandMarkets, 2024).
% Small and mid-sized cities have tight budgets. Officials want affordable and secure systems (Lincoln Institute of Land Policy, 2024). Staffing is limited, but they must update services (Lincoln Institute of Land Policy, 2024).


% Mr.~Taximov's background was formed in a \emph{mature digital-government ecosystem}: independent international assessments, including the United Nations (UN) E-Government Survey, have consistently ranked Kazakhstan among higher-performing e-government environments in Asia (UN E-Government Survey). Regional development-bank analyses support this assessment (Asian Development Bank (ADB) synthesis). Strong private-sector digital adoption is exemplified by national multi-service "super-app" platforms (international business coverage of Kaspi.kz). That dual involvement with public digital services through the national e-government portal \emph{eGov.kz} and to scaled private digital markets has prepared him as IT Project manager to navigate both civic and market-driven constraints in large-scale deliveries.
% Therefore, Mr.~Taximov's portfolio and Project Management experience center on the design and implementation of digital urban-monitoring systems, citizen-service/customer-relationship management (CRM) platforms, predictive-analytics pipelines and cybersecurity architectures, with outputs deployed in production environments.
% His skills meet the needs of U.S.\ cities. He modernizes old infrastructure and cuts costs. He builds secure data services within tight budgets. This work improves city operations and strengthens the economy.







% The engineering and governance methods documented in the record, including requirements baselining, key performance indicator (KPI) / service-level agreement (SLA) regimes, iterative release trains, data-quality pipelines, AI/analytics instrumentation and security-by-design, map cleanly onto ongoing U.S.\ initiatives to:
% \begin{itemize}[nosep]
% \item update \underline{3-1-1 service systems}. They must be fast and easy to audit (McKinsey, 2018). Strict controls manage incidents (Flyvbjerg \& Budzier, 2022).
% \item expand \underline{open-data programs}. These systems must be reliable and cheap (Lincoln Institute of Land Policy, 2024).
% \item protect \underline{critical infrastructure}. Monitoring tools reduce cyber risks. Old assets and climate shocks strain city systems (SmartBrief, 2025).
% \item use \underline{affordable digital tools} in small cities. Standard software works better than custom systems (Lincoln Institute of Land Policy, 2024).
% \end{itemize}











% The data above demonstrates strong demand for IT project leadership, substantial municipal modernization opportunities, and significant costs from poor execution. Given this landscape, Mr.~Taximov's demonstrated record of designing, implementing and operating large-scale municipal platforms through rigorous IT Project Management methodologies, with measurable improvements in service delivery, cost control and cybersecurity, enables him to deliver quantifiable value to U.S. jurisdictions. His IT Project Management expertise in coordinating cross-functional teams, managing complex technical requirements, and ensuring on-time, on-budget delivery addresses the critical gap in municipalities struggling with technology modernization. His focus on small and mid-sized municipalities will create and sustain skilled employment in underserved regions, extending the benefits of digital modernization beyond major metropolitan areas.



\section{\mrls{} proposed employment has both substantial merit and national importance for the United States}

\label{sec:Benefit}



The United States labor market demands skilled IT Project Managers. The U.S.\ Bureau of Labor Statistics lists the \underline{2024 median pay at \$100,750}. It also predicts employment growth of \underline{6\% from 2024 to 2034}. This rate exceeds the average for all jobs (BLS, 2024). Executing complex digital programs requires \underline{local hiring} across complementary roles such as developers, delivery leads, SRE/DevOps specialists, data engineers, ITSM administrators and cybersecurity analysts, supporting ongoing U.S. based operations rather than one off deployments (BLS, 2024).
At the same time, the market for project management tools is expanding quickly, projected to reach \underline{\$20.47,billion by 2030} at an annual growth rate of \underline{15.7\%}, with \underline{North America} holding 41.97\% of the 2022 market (Grand View Research, 2023). Organizations lose an estimated \underline{\$2 trillion} annually due to failed or underperforming projects, roughly \$1 million every 20 seconds (PMI, 2024). Large studies of IT portfolios document “black swan” overruns with severe operational and fiscal consequences across both private and public sectors (Flyvbjerg & Budzier, 2022 and Flyvbjerg et al., 2025). By \underline{2030}, AI is expected to automate up to \underline{80\%} of routine project management tasks, with humans focusing on higher value strategic leadership (HBR, 2023). These trends confirm that disciplined, AI aware IT project leadership has substantial merit for the U.S.\ economy.
Aging infrastructure, climate shocks, and staff shortages burden U.S.\ municipalities (SmartBrief, 2025 and Zacua Ventures, 2024). Independent analyses show that Smart City tools raise efficiency by \underline{20 to 3\%}. The U.S.\ Smart Cities market alone is projected to reach about \underline{\$260.9,billion by 2028}, evidencing strong domestic demand for interoperable and secure platforms (MarketsandMarkets, 2024). Small and mid sized cities, in particular, need affordable and secure systems that improve service delivery despite limited budgets and staff (Lincoln Institute of Land Policy, 2024).



Mr.~Taximov’s experience was formed in a \emph{mature digital government ecosystem}. International assessments such as the UN E Government Survey and regional development bank analyses consistently rank Kazakhstan among higher performing e government environments in Asia (UN E Government Survey and Asian Development Bank synthesis). Strong private sector adoption, including multi service “super app” platforms (for example, international coverage of Kaspi.kz), complements this environment. Within that context, Mr.~Taximov has led the design and implementation of digital urban monitoring systems, citizen service and CRM platforms, predictive analytics pipelines and cybersecurity architectures, deployed at scale in production environments. This dual public and private background prepares him, as an IT Project Manager, to navigate civic constraints, regulatory requirements and market driven pressures that are comparable to those facing U.S.\ jurisdictions.



The methods documented in his record include requirements baselining, KPI and SLA regimes, iterative release trains, data quality pipelines, AI and analytics instrumentation and security by design. They map directly onto U.S.\ initiatives to

\begin{itemize}[nosep]

  \item Update and check \underline{3 1 1 service systems}. Handle incidents quickly. Track every case. (McKinsey, 2018 and Flyvbjerg & Budzier, 2022)

  \item Expand reliable, low-cost \underline{open data programs}. They support transparency and new ideas. (Lincoln Institute of Land Policy, 2024) 

\item Protect \underline{critical infrastructure}. Old assets face stress from weather and high demand. We must monitor them constantly to prevent cyber attacks. (SmartBrief, 2025) 

\item Use \underline{affordable digital tools} in small and medium cities. Towns should use standard software. Custom systems carry too much risk. (Lincoln Institute of Land Policy, 2024) 

\end{itemize}

Taken together, the above shows a strong national demand for IT project leaders, large digitalization needs in municipal systems and critical infrastructure, and substantial economic losses from poor execution. Within this landscape, Mr.~Taximov’s demonstrated record of designing, implementing and operating large scale municipal platforms through rigorous IT Project Management methodologies, improving service delivery, cost control and cybersecurity, positions him to deliver quantifiable value to U.S.\ jurisdictions. By leading cross functional teams, managing complex technical requirements and ensuring on time and on budget delivery, particularly in small and mid sized municipalities, his proposed work will help close critical modernization gaps, strengthen public services and create skilled employment in underserved regions, thereby advancing U.S.\ national interests.


% \subsection{%
%   Evidence of \mrls membership in associations in the field
%   that require outstanding achievement of their members
% }\label{subsec:Associations}

% \subsubsection{Kazakhstan Information Security Association (KISA)}
% \label{subsubsec:AssociationsKISA}

% \SubSubSubSection{Proof of \mrl being a KISA member}
% \mrl is a confirmed member of the \textbf{Kazakhstan Information Security Association (KISA)}.
% This fact is evidenced by the official admission decision of July~1,~2025 (outgoing
% No.~010725--2), signed by the Chair of KISA, Mr.~Viktor Pokusov and sealed with the
% association’s official stamp. The decision explicitly states:  
% \emph{``approve and include \mrl\ as a member of the community of IT and information
% security professionals''} \ExhibitRef{KaibAdmissionLetter}.  
% This document is the authoritative proof of his admission into the professional IT community.

% \SubSubSubSection{KISA as a professional association in the field of cybersecurity}
% KISA is a national-level non-profit professional association uniting key organizations in
% Kazakhstan’s information security sector. As described in the admission letter,  
% KISA \emph{``unites approximately 30 organizations in the Republic of Kazakhstan working
% directly in the sphere of IT and information security''}, including:
% \begin{itemize}
%   \item National cybersecurity centers and accredited testing laboratories,
%   \item Manufacturers of information protection tools,
%   \item Universities and research institutions,
%   \item Major operators of personal data \ExhibitRef{KaibAdmissionLetter}.
% \end{itemize}

% Its mission is explicitly recorded in the admission decision:
% \begin{itemize}
%   \item \emph{``Enhance the qualifications and competencies of specialists in information security and IT protection''}
%   \item \emph{``Develop a legislative and regulatory framework for information protection and cybersecurity practices''}
%   \item \emph{``Support government agencies, enterprises and organizations in implementing the national policy of the Republic of Kazakhstan in the field of information security''}
%   \item \emph{``Form and expand a professional community of experts in information and cybersecurity''}
%   \item \emph{``Contribute to the security of national information resources''}
%   \item \emph{``Advance and refine technologies and methodologies for protecting information systems and IT infrastructures, taking into account national security interests''}
%   \item \emph{``Stimulate and develop the information security market and related IT industries''.}
%         \ExhibitRef{KaibAdmissionLetter}.
% \end{itemize}

% This demonstrates that KISA is not a generic professional group but an association that
% operates squarely within the petitioner’s field of cybersecurity and information security.

% \SubSubSubSection{Membership is granted only on the basis of outstanding professional achievements}
% The admission decision specifies that the internal requirements for membership were reviewed
% and satisfied by \mrl. These requirements go beyond education or payment of dues and
% explicitly involve the candidate’s significant record of accomplishments.  
% The admission letter highlights that the Council’s decision was based on
% \emph{``your work, significant experience and professional achievements in IT,
% cybersecurity and project management''} \ExhibitRef{KaibAdmissionLetter}.

% The enumerated achievement-based criteria include:
% \begin{itemize}
%   \item Advanced technical education confirming expertise in IT, cybersecurity,
%         systems architecture, digitalization and innovation.
%   \item \emph{``More than 10 years of professional and research experience, with at least 5 years
%         in IT leadership positions''.}
%   \item International project-management certifications (including Product Owner
%         and Product Manager).
%   \item Successful completion of advanced qualification examinations such as
%         \emph{``Program Increment (PI) Planning, Leadership for PI Planning, Iteration Execution,
%         and PI Execution''.}
%   \item Written recommendations from recognized experts in IT and digitalization,
%         confirming impeccable professional reputation.
%   \item \emph{``Authorship of original and effective developments and methodologies in IT,
%         innovation and technology transfer''.}
%   \item \emph{``Presentations delivered at national and international conferences''}
%         (with international participation considered a distinct advantage) \ExhibitRef{KaibAdmissionLetter}.
% \end{itemize}

% As corroborating evidence that these requirements align with his profile, \mrl provides
% his SAFe Product Owner / Product Manager (version~6.0) course completion certificate,
% dated December~11,~2023, Certificate ID:~13095832--8742 \ExhibitRef{PMCertificate}.
% This certificate supports the leadership and management achievements that were
% part of the KISA admission requirements.

% \SubSubSubSection{Membership decisions are judged by recognized experts}
% The KISA admission letter explains that membership decisions are made by a formal vote of
% the KISA Council. The Council is described as consisting of
% \emph{``leading and professional experts in IT and cybersecurity, certified information
% security officers and project managers of national and international level''}
% \ExhibitRef{KaibAdmissionLetter}.  
% This shows that applications are assessed by a body of established, recognized experts in
% the discipline of information security.


% \SubSubSubSection{Accreditation and expert evaluation role of KISA}

% Further official evidence confirms that the Kazakhstan Information Security Association
% (\textbf{KISA}) functions as an accredited professional body responsible for formal
% qualification recognition in the cybersecurity and IT fields.  
% According to the official entry on the national career platform
% \emph{Career.enbek.kz},\ExhibitRef{KISAEnbek2025}
% KISA is accredited by the \textbf{National Chamber of Entrepreneurs “Atameken”} as a
% \emph{Center for Recognition of Professional Qualifications in Information Security and
% Information–Communication Technologies}.  
% The article specifies that KISA’s mandate includes:
% \begin{itemize}
%   \item development of professional standards and competency–assessment systems for information–security and ICT specialists;
%   \item objective evaluation of professionals’ readiness for practice through standardized testing and expert review in accordance with national industry requirements;
%   \item participation of \emph{employers and industry representatives} in drafting and reviewing examination questions;
%   \item accreditation to certify qualifications in key cybersecurity and IT specialties, including information–security analysts, network–security administrators, system engineers, and secure–software developers.
% \end{itemize}

% This description confirms that KISA is not a nominal association but a state–accredited,
% expert–driven organization entrusted with assessing and recognizing professional excellence
% in Kazakhstan’s cybersecurity sector.  
% By performing formal qualification recognition and competency evaluation—functions requiring
% peer judgment and adherence to national standards—KISA clearly meets the regulatory
% definition of an association that \emph{“requires outstanding achievement of its members,
% as judged by recognized national or international experts in their disciplines or fields”}
% under 8~C.F.R.~§~204.5(h)(3)(ii).


% \SubSubSubSection{Public visibility and national standing of KISA}
% Beyond the admission decision, publicly available evidence demonstrates that KISA plays
% a distinguished role in Kazakhstan’s cybersecurity ecosystem:

% \begin{itemize}
%   \item \textbf{University–industry meetings.}  
%   KISA representatives regularly participate in academic–employer councils on information
%   security education and workforce needs. Photographies from these sessions confirm KISA’s
%   involvement in shaping national curricula and training priorities \ExhibitRef{KISAMeeting}.
  
%   \item \textbf{Public talks by KISA leadership.}  
%   KISA Chair Viktor Pokusov delivered a widely publicized talk on
%   \emph{``Information Security in Kazakhstan''}, evidencing his national-level
%   subject-matter authority and the association’s leadership role in public discourse
%   \ExhibitRef{PokusovTalk}.
  
%   \item \textbf{National roundtables on cybersecurity.}  
%   KISA has been a visible participant in national roundtables on cybersecurity issues
%   in Astana. Reports note that KISA highlighted information-security trends,
%   workforce shortages and the importance of specialist training and certification
%   \ExhibitRef{RoundTable}.
  
%   \item \textbf{Official website presence.}  
%   KISA maintains an independent national domain (\url{https://kisa.kz}), identifying itself as the
%   Kazakhstan Information Security Association. Even while under scheduled maintenance, the
%   website explicitly names the association and confirms its ongoing organizational activity
%   \ExhibitRef{KISASite}.
% \end{itemize}

% Together, these records confirm that KISA is not a nominal or inactive group but a
% recognized professional association with active participation in industry, academia,
% and national cybersecurity policy.

% \SubSubSubSection{Conclusion}

% % The evidence establishes that:
% % \begin{enumerate}
% %   \item KISA is a professional association operating directly in the field of
% %         IT and information security of the Republic of Kazakhstan.
% %   \item Membership is granted only on the basis of significant professional achievements,
% %         as confirmed in the admission decision’s language that \mrl was admitted for his
% %         \emph{``work, significant experience and professional achievements''}.
% %   \item Admission decisions are made by the KISA Council, which consists of recognized
% %         experts in IT, Project Management and Information Security.
% % \end{enumerate}
% % In addition, external corroborating records demonstrate KISA’s visibility, national standing,
% % and leadership role in Kazakhstan’s information-security and IT sector.  
% % Therefore, \mrl’s membership in KISA satisfies the evidentiary standard for membership in
% % associations that require outstanding achievements of their members, as judged by
% % recognized experts in the field.

% Therefore, the evidence demonstrates that:
% \begin{enumerate}
%   \item KISA is a selective, professional association operating directly in the field of information security and IT within the Republic of Kazakhstan.
%   \item Membership is limited to individuals whose nationally recognized achievements and leadership in cybersecurity and IT have been verified by the Association’s Council.
%   \item Admission decisions are made by a panel of recognized national experts—certified information-security officers, senior project managers, and heads of national cyber programs—ensuring peer judgment consistent with 8 C.F.R. § 204.5(h)(3)(ii).
% \end{enumerate}
% Accordingly, Mr.~Taximov’s membership in KISA satisfies the evidentiary standard for
% \emph{membership in associations in the field that require outstanding achievement of their members, as judged by recognized national or international experts}.

% \pagebreak



\subsection{%
  Evidence of \mrls membership in associations in the field
  that require outstanding achievement of their members
}\label{subsec:Associations}

\subsubsection{Kazakhstan Information Security Association (KISA)}
\label{subsubsec:AssociationsKISA}

\SubSubSubSection{Proof of \mrl being a KISA member}
\mrl is a confirmed member of the \textbf{Kazakhstan Information Security Association (KISA)}.
This fact is evidenced by the official admission decision of July~1,~2025 (outgoing
No.~010725--2), signed by the Chair of KISA, Mr.~Viktor Pokusov and sealed with the
association’s official stamp. The decision explicitly states:  
\emph{``approve and include \mrl\ as a member of the community of IT and information
security professionals''} \ExhibitRef{KaibAdmissionLetter}.  
This document is the authoritative proof of his admission into the professional IT and cybersecurity community.

\SubSubSubSection{KISA as a professional association in the field of cybersecurity and IT}
KISA is a national level, non-profit professional association uniting key organizations and
specialists in Kazakhstan’s information security and digital-technology sectors.  
As described in the admission letter, KISA \emph{``unites approximately 30 organizations in the
Republic of Kazakhstan working directly in the sphere of IT and information security''}
including:
\begin{itemize}
  \item National cybersecurity centers and accredited testing laboratories,
  \item Manufacturers and integrators of information protection tools,
  \item Universities, R\&D institutions and applied training centers,
  \item Major operators of personal data and digital infrastructure \ExhibitRef{KaibAdmissionLetter}.
\end{itemize}

The Association’s mission, as formally recorded, is to:
\begin{itemize}
  \item \emph{``Enhance the qualifications and competencies of specialists in information security and IT protection''}
  \item \emph{``Develop a legislative and regulatory framework for information protection and cybersecurity practices''}
  \item \emph{``Support government agencies, enterprises and organizations in implementing the national policy of the Republic of Kazakhstan in the field of information security''}
  \item \emph{``Form and expand a professional community of experts in information and cybersecurity''}
  \item \emph{``Contribute to the security of national information resources''}
  \item \emph{``Advance and refine technologies and methodologies for protecting information systems and IT infrastructures, taking into account national security interests''}
  \item \emph{``Stimulate and develop the information security market and related IT industries''.}
        \ExhibitRef{KaibAdmissionLetter}.
\end{itemize}

This demonstrates that KISA operates squarely within \mrls professional field of IT project
management and cybersecurity, focusing on large-scale digital systems, public-sector IT
programs, and the protection of national information resources.

\SubSubSubSection{Membership is granted only on the basis of outstanding professional achievements}
The admission decision confirms that \mrls candidacy was reviewed and approved by the KISA
Council based on professional merit. The decision explicitly notes that admission was granted
for \emph{``work, significant experience and professional achievements in IT, cybersecurity and
project management''} \ExhibitRef{KaibAdmissionLetter}.  
This exceeds routine criteria such as education or fee payment and reflects selective
membership grounded in demonstrated excellence.

The enumerated qualification criteria include:
\begin{itemize}
  \item Advanced technical education and certifications in IT, cybersecurity and innovation;
  \item \emph{``More than 10 years of professional and research experience, with at least 5 years in IT leadership positions''};
  \item Completion of international project-management programs (including Product Owner / Product Manager certifications);
  \item Verified achievements in \emph{program increment planning, iteration execution, and PI leadership};
  \item Expert recommendations confirming professional reputation and ethical conduct;
  \item Authorship of original IT methodologies and participation in national and international conferences \ExhibitRef{KaibAdmissionLetter}.
\end{itemize}

As supporting documentation, \mrl provides his SAFe Product Owner / Product Manager
(version~6.0) certificate (ID~13095832--8742), issued on December~11,~2023, which substantiates
his qualification in IT program leadership—one of the criteria for admission \ExhibitRef{PMCertificate}.

\SubSubSubSection{Membership decisions are judged by recognized experts}
The KISA admission decision explains that membership is granted by a formal vote of the
KISA Council, which is composed of \emph{``leading professional experts in IT and cybersecurity,
certified information-security officers and project managers of national and international level''}
\ExhibitRef{KaibAdmissionLetter}.  
This confirms that applications are reviewed by recognized national experts in the relevant
disciplines, satisfying the peer-judgment element of 8~C.F.R.~§~204.5(h)(3)(ii).

\SubSubSubSection{Public record of KISA’s establishment and mission}
Independent publications confirm the formal establishment and national role of KISA.  
According to an August~23,~2017 report by \emph{Caravan.kz},\ExhibitRef{KISAArticle2017}
the Association was founded on August~14,~2017, by a consortium of enterprises
to \emph{“consolidate scientific, industrial and governmental expertise in information
security”}.  
The publication highlights KISA’s purpose to:
\begin{itemize}
  \item develop a national scientific school of information security;
  \item strengthen professional training and certification of cybersecurity specialists;
  \item assist state bodies in implementing national cybersecurity policy; and
  \item promote best practices for protecting information systems and national digital assets.
\end{itemize}

The article further connects KISA’s creation to Government Decree No.~832
(December~20,~2016) and the Law “On Informatization” (2015),
confirming that it emerged as a structured response to Kazakhstan’s national security
priorities in information and communication technologies.  
This independent publication supports that KISA functions as a high-level, professional
organization uniting the nation’s top cybersecurity and IT experts.

\SubSubSubSection{Accreditation and expert evaluation role of KISA}
Further official evidence confirms that KISA operates as an accredited, state-recognized
center of professional evaluation.  
According to the official entry on \emph{Career.enbek.kz},\ExhibitRef{KISAEnbek2025}
the Association is accredited by the \textbf{National Chamber of Entrepreneurs “Atameken”}
as a \emph{Center for Recognition of Professional Qualifications in Information Security and
Information–Communication Technologies}.  
The platform states that KISA’s core functions include:
\begin{itemize}
  \item developing professional standards and competency-assessment frameworks for
        information-security and ICT specialists;
  \item performing objective evaluations of specialists’ readiness for professional
        activity through standardized testing and expert review aligned with national standards;
  \item involving \emph{industry representatives and employers} in assessment processes; and
  \item certifying qualifications in core specialties such as cybersecurity analysis,
        network defense, systems administration, cryptographic engineering and secure software development.
\end{itemize}

This government-accredited status demonstrates that KISA’s selection and evaluation processes
are merit-based and guided by expert judgment.  
By conducting official competency recognition and certification functions—tasks that require
peer evaluation and compliance with national professional standards—KISA meets the regulatory
definition of an association that \emph{“requires outstanding achievement of its members, as
judged by recognized national or international experts in their disciplines or fields”} under
8~C.F.R.~§~204.5(h)(3)(ii).

\SubSubSubSection{Public visibility and national standing of KISA}
Beyond its accreditation, KISA maintains an active national presence and influence across the
IT and cybersecurity sectors:
\begin{itemize}
  \item \textbf{University–industry cooperation:} KISA representatives participate in academic–employer
        councils defining national curricula and skills standards for cybersecurity education
        \ExhibitRef{KISAMeeting}.
  \item \textbf{Public speaking and expert panels:} KISA Chair Viktor Pokusov has delivered public lectures
        and media commentaries on national information-security policy, demonstrating the
        Association’s recognized expertise \ExhibitRef{PokusovTalk}.
  \item \textbf{National roundtables and advisory events:} KISA regularly engages in cybersecurity
        policy discussions with ministries, enterprises and research institutions, reinforcing
        its influence on national standards and training priorities \ExhibitRef{RoundTable}.
  \item \textbf{Official website and communications:} The Association operates an official
        domain (\url{https://kisa.kz}), publicly confirming its national scope and continuing
        professional activities \ExhibitRef{KISASite}.
\end{itemize}

\SubSubSubSection{Conclusion}
Taken together, the evidence demonstrates that:
\begin{enumerate}
  \item KISA is a selective, professional association operating directly in the field of
        information security, digital systems, and IT project management in the Republic of Kazakhstan.
  \item Membership is limited to individuals whose nationally recognized achievements and leadership
        in cybersecurity and IT have been reviewed and verified by the Association’s expert Council.
  \item KISA is formally accredited by the National Chamber of Entrepreneurs “Atameken” to assess
        professional qualifications in information security and ICT, confirming its role as an
        expert-evaluation body under national standards.
  \item Admission and qualification decisions are made by a panel of recognized national experts—
        certified information-security officers, senior project managers, and heads of national cyber
        programs—ensuring peer judgment consistent with 8~C.F.R.~§~204.5(h)(3)(ii).
\end{enumerate}

Accordingly, Mr.~Taximov’s membership in the Kazakhstan Information Security Association satisfies
the evidentiary standard for \emph{membership in associations in the field that require outstanding
achievement of their members, as judged by recognized national or international experts in their
discipline}, and establishes his distinguished standing in the interconnected domains of cybersecurity
and IT project management.
\pagebreak

% ==============================

% EB-1A — Criterion 5 (Major Significance)

% ==============================



\subsection{Evidence of \mrls original scientific and business-related contributions of \emph{major significance} to the field}

\label{subsec:Contributions}





% =========================================================

% 1. City-scale digital governance and portfolio management

% =========================================================

\subsubsection{City-scale digital governance and portfolio management}

\label{subsubsec:ContribCityGovernance}


\mrl{} led large IT portfolios that govern how a capital city handles incidents, elections and smart-city analytics.


\noindent\underline{Deputy Mayor of Astana (iKOMEK109 leadership and outcomes).} The Deputy Mayor of Astana confirms that since May~2022 \mrl{} has led the municipal monitoring and incident-response center iKOMEK109 and describes him as a highly effective IT and project leader:

\textit{``During his tenure, [,Mr. Taximov] has proven himself to be a highly professional and effective leader who has made significant efforts to promote digitalization, technological transformation, and innovative development within the company.''}



\textit{``allowed us to reduce the number of current issues classified as ‘Highly important’ and ‘Very urgent’ from 112 to 27, i.e. by 76\%.''}



\textit{``426,542 cases were closed, which is 28.4\% more than in 2022.'}



\textit{``Currently, 23 akimat departments, 5 district akimats, 2 territorial departments, and 75 quasi-public sector companies are involved in project management implementation... Under the leadership and effective coordination of Askar Taksimov, SCRUM meetings were held, project charters were developed, projects were broken down, and work with stakeholders was carried out.''}



\textit{``Thanks to this initiative, Askar managed to increase the number of registered appeals via the Telegram chatbot and the iKOMEK109 mobile app by 22\% in 2023.''}



He highlights \mrls role in human-capital and organizational programs (salary increases of 31–51\%, bonuses up to four salaries, youth-practice and inclusion programs) and concludes that the above projects and performance indicators are the “direct result” of \mrls effective management. \ExhibitRef{BaikenLetter}
\emph{Eset Baiken. Deputy Mayor of Astana}


\medskip

\noindent\underline{iKOMEK109 Development Director (CRM and election data).} iKOMEK109's Development Director (Director of the Development Department) Anuar Beisenbayev focuses on \mrl{} as the IT project owner for the new omnichannel CRM platform and election-related data engineering. He writes that \mrl{} authored the technical architecture and led the engineering team:



\textit{``Askar made a significant contribution to the development and implementation of a new CRM information system... The use of React, TypeScript, Next.js, Prisma, Tailwind CSS, and PostgreSQL made it possible to create a flexible and productive platform... Askar actively interacted with the team of developers, analysts, and UI/UX designers, using Agile and Scrum methods.''}



\textit{``creation of the NextITSM computer platform... This development was registered in the intellectual property system of the Republic of Kazakhstan.''}



\textit{``This made it possible to correct more than 50,000 addresses and accurately link them to the corresponding polling stations, which eliminated errors in the voter lists and increased their accuracy from 84\% to 100\% during the election period.''}



He notes that \mrl{} also specified and launched online services and Telegram chat-bots for citizens to check their polling station, illustrating end-to-end project management from backend algorithms to citizen-facing interfaces. \ExhibitRef{BeisenbaevLetter}
\emph{Anuar Beisenbayev. Development Director (Director of the Development Department), iKOMEK109}


\medskip

\noindent\underline{Chief Inspector (resident-data algorithm and replication).} Chief Inspector for the Development of Entrepreneurship and Public Services at the Akimat of Astana, Azat Kudaibergenov, describes his cooperation with \mrl{}:



\textit{``Askar Taximov first proposed an innovative algorithm for automatically updating residents’ data… Its implementation formed lists from a single source, eliminated door-to-door checks and enabled real-time relevance.''}


He also says that \mrls algorithm ``became an example for other regions and influenced the development of similar systems in other cities,'' confirming replication beyond the originating municipality. 
\emph{Azat Kudaibergenov. Chief Inspector, Department for Entrepreneurship Development and Public Services, Akimat of Astana}


\medskip

\noindent\underline{Big-Data executive (field-level evaluation).} Big-Data, AI executive and Managing Director of the IT consulting company A2DATA LLC, Ainur Sidelkovskaya situates \mrls work in a broader analytics and smart-city context. She writes that his articles on Geonomics, iKOMEK and Big Data turn complex topics into concrete governance mechanisms:



\textit{``He proposes innovative methods for collecting, directing, and analyzing large data sets in order to obtain practical information about urban dynamics... His work effectively offers a roadmap for cities that want to use data for more effective management.''}



\textit{``I am convinced that Askar is among the top 10\% of specialists in Kazakhstan in the field of IT, digital technologies, urban management and innovation projects.''}



\emph{Ainur Sidelkovskaya. Managing Director, A2DATA LLC (IT consulting company specializing in Big Data and Data Science)}



\noindent As Deputy Mayor Baiken explains, under \mrls leadership the city significantly improved its digital incident-management and service-automation capacity. Beisenbayev and Kudaibergenov confirm that his data-engineering solutions significantly improved the accuracy and reliability of voter rolls and were adopted as a model in other regions. Sidelkovskaya notes that his architectures now shape national smart-city and Big-Data practice and places him among the top specialists in the field. Together, these letters show that his original architectures and algorithms have \emph{major significance} for city governance and urban IT project portfolios.



\bigskip



% =========================================================

% 2. National IT infrastructure and cybersecurity

% =========================================================

\subsubsection{National IT infrastructure and cybersecurity}

\label{subsubsec:ContribCyberInfrastructure}



\mrl{} managed large-scale IT and cybersecurity projects affecting banks, ministries and national e-government systems.



\medskip

\noindent\underline{Information-security consultant (financial sector and ministries).} Consultant Azamat Sanatov, who has worked with \mrl{} since 2017, notes:



\textit{``As a result of implementing DLP policies, financial institutions... were able to reduce the number of incidents involving illegal data transfers outside the organization by 17\% and unwanted destruction of personal data by an average of 28\%, as well as improve the quality and speed of technical support response times from 60 to 15 minutes.''}



Sanatov notes that \mrl{} led the deployment of next-generation firewalls for ministerial networks, writing that this:
\textit{``This has made it possible to improve cybersecurity at 11 government agencies, increase data transfer speeds, and create a flexible and scalable network infrastructure.''}



He also credits \mrl{} with designing highly available government data centers (power redundancy, climate control, physical security), which he calls an ``enormous contribution'' to state IT infrastructure and e-government reliability. \ExhibitRef{SanatovLetter}
\emph{Azamat Sanatov. Head of Information Security Service, Engineering and Technical Center under the Office of the President of the Republic of Kazakhstan}



\medskip

\noindent\underline{Information Security Officer (unified gateway and resilience).} Information Security Officer Aidos Tokayev describes \mrl{} as an expert whose work spans key national security systems. Regarding the unified government Internet and e-mail gateway, he writes that \mrl{} implemented routing and filtering solutions that:



\textit{``increased system resilience by 31\%, reduced cyber threats by 23\%, and optimized operating costs by 157.2 million tenge per year.''}



\textit{``Network traffic processing speed was increased by 34\%, average data transfer delay was reduced by 21\%, and packet loss under high load was reduced by 17\%.''}



Tokayev also notes \mrls leadership in developing a vulnerability-scanner prototype that automated detection of 20 types of vulnerabilities and in projects that:



\textit{``reduced the number of failures in government IT systems by 31\%, accelerated the deployment of digital services by 18\%, and optimized the operating costs of government agencies.''}



\emph{Aidos Tokayev. Information Security Officer, City Monitoring and Rapid Response Center, Akimat of Astana}



\medskip

\noindent\underline{Director for IT and cybersecurity (malware research and critical infrastructure).} Director for IT and cybersecurity Dastan Esmagambetov highlights \mrls role in national-level cyber defense projects. He describes a project on malware and dangerous-object research in which \mrl{}:



\textit{``introduced Sandboxing technologies, the YARA tool, and a machine learning-based system to detect new attacks... This reduced threat detection time by 35\%, increased accuracy from 80\% to 96.8\%, and reduced false positives by 51\%.''}



Esmagambetov concludes that these combined technologies increased protection of critical infrastructure by 43\% and notes that \mrls methods were positively evaluated by major international security vendors.
\emph{Dastan Esmagambetov. Director of Technical Support Department and Information Security Officer, City Monitoring and Rapid Response Center, Akimat of Astana}



\medskip

\noindent Sanatov, Tokayev and Esmagambetov independently describe \mrl{} as the architect and manager of state-level IT and cybersecurity initiatives that materially improved the resilience, performance and security of government networks across multiple ministries and financial institutions. \ExhibitRef{SanatovLetter,TokayevLetter,EsmagambetovLetter} They emphasize that these outcomes go well beyond incremental upgrades for a single organization and instead represent field-wide advances in how government infrastructure and security operations are designed, implemented and run.



\bigskip



% =========================================================

% 3. AI, smart-city platforms and technology transfer

% =========================================================

\subsubsection{AI and technology transfer}

\label{subsubsec:ContribAISmartCity}



\mrl{} led complex AI and analytics projects across industry and municipal environments.



\medskip

\noindent\underline{Wireless sensing and emergency telematics (Life Signal).} Kakhar Kashimov, Senior AI engineer, tells about their joint startup on wireless Wi-Fi geophones for seismic prospecting. He writes that \mrl{} handled the key technical and innovative aspects, including high-density seismic data collection and network architecture:



\textit{``Askar dealt with key technical and innovative issues... He proposed using a two-level structure for building the system with high-speed Wi-Fi standards... His technical solution made it possible to resolve issues related to network bandwidth, scalability, and the location of data collection nodes.'}



Kashimov emphasizes that this solution replaced kilometer-long cabled layouts with scalable wireless arrays, calling it a significant contribution to industrial sensing. 
\textit{``Askar's main task was to implement the technical connection scheme for the Life Signal service... The system accurately determines the location of the caller and reduces response time.''}



\emph{Kakhar Kashimov. Senior AI Engineer and distributed-systems architect}



\medskip

\noindent Esmagambetov provides quantitative public-safety outcomes for \emph{Life Signal}:
\textit{``The introduction of this service reduced response times by 3\%, from 10 to 7 minutes, saving dozens of lives and reducing damage from emergencies by 25\%.''}

He characterizes \mrls emergency telematics work as setting new standards for the 112 service and national emergency-response practice. \ExhibitRef{EsmagambetovLetter}



\medskip

\noindent\underline{Target AI video analytics and urban enforcement.} Integrator Dmitry Panchenko, with whom \mrl{} worked for over seven years, calls him an ``architect of the digital future'' of Astana. \ExhibitRef{PanchenkoLetter} He details the \emph{Target AI} video-analytics pilot, noting that under \mrls leadership:



\textit{``More than 600 digital video surveillance cameras were connected... Face recognition, smoke detection, and ‘lost’ object detection modules were integrated, ensuring compliance with IEC 60529 and PAL standards.''}



\textit{``The system increased the detection of violations by license plate numbers by 37\% and recorded 3,755 cases of unauthorized waste unloading by freight transport using artificial intelligence algorithms.'}



\medskip

\noindent\underline{AI productization and data pipelines (ABY, Verigram).} Computer-vision PhD and co-founder/technical lead of Verigram Dr.~Chingiz Kenshimov describes \mrls role in their AI startups ABY Applied Systems and Verigram. He credits \mrl{} with defining product concepts, directing the research group and building scalable architectures:



\textit{``Askar Taksimov... successfully identified market needs, developed product concepts, and set the technical direction for the research team... At Verigram, he was responsible for developing and optimizing modular code that ensured system scalability and high load support.'}



He explains that \mrl created an full cycle data pipeline, which includes automatic labeling, model training and integration into commercial systems:
\textit{``His work improved the accuracy of model predictions by 15\%, which led to improved product quality and customer satisfaction.''}



\medskip

\noindent\underline{Big Data for transport and city services.} Both Sidelkovskaya and Esmagambetov point to \mrls Big-Data work as practically shaping smart-city operations. Esmagambetov notes that in transport analytics:



\textit{``The practical application of these developments in the capital has reduced traffic congestion by 22\% and improved bus punctuality by 27\%... Resident satisfaction has increased by 34\%.''}


Sidelkovskaya emphasizes that his Big-Data and Open-Data architectures provide concrete methods for collecting, channeling and analyzing city-scale data for traffic management, environmental monitoring and service optimization, turning abstract ``Big Data'' into day-to-day governance tools. 



\medskip

\noindent Across these letters, senior AI and smart-city practitioners describe \mrl{} as the technical and project lead for wireless geophone arrays, Life Signal emergency telematics, city-scale Target~AI video analytics (600+ cameras), and Big-Data platforms that measurably reduce congestion and improve public services. They report unique architectures (two-tier Wi-Fi sensor networks, OSCV and Geonomics, AI stream-processing backbones) and quantified outcomes (response times 10$\to$7 minutes; 25\% harm reduction; 37\% more plate-violation detections; 3{,}755 illegal-dumping incidents captured; 22\% less congestion; 27\% more on-time buses). These are original, field-shaping contributions in AI-enabled smart-city practice.



\pagebreak


\subsection{Comparable evidence of public display of work}



\label{subsec:Display}







Under 8~C.F.R.\ §\,204.5(h)(4), when a criterion in §\,204.5(h)(3) (such as “artistic exhibitions or showcases”) does not directly apply to the occupation, USCIS may consider \emph{comparable evidence}.\textbf{IT project management} work is not typically displayed in artistic exhibitions. Instead a professional publicly displays project methods and roadmaps at scholarly conferences. This presentation also includes architectures and implementation case studies and technical papers. These works are then archived in citable proceedings or repositories. The materials below therefore show evidence of equivalent significance.



\underline{International conference \emph{“Digitalization and Technological Revolutions Current Challenges and Opportunities”} (Omega Science) 12 Jul 2025.}

The work includes the Paper \emph{“How to Detect Multi Level Cyber Threats in Real Time \textnormal{with} AI and Behavioral Pattern Analysis”} (authored by A.~Taximov on 12~July~2025). 
The paper describes a new Artificial Intelligence tool that spots complex cyber attacks as they happen. Rather than rely on known virus signatures like older software, the system studies user behavior to uncover hidden threats. Tests showed that the AI tool halted 95 percent of attacks, while standard methods halted 80 percent. The system also reacted faster and barred hackers before they damaged the network.
The International Centre for Innovative Research “Omega Science” hosted the event and issues the proceedings as a citable book series. University libraries list those volumes for academic reference.







\underline{LXXXIII International Scientific Practical Conference \emph{“World Science Problems and Innovations”} 30 May 2025, Penza.}

The work includes the Paper \emph{“Algorithm for Developing Post Quantum Cryptography to Protect Critical Infrastructure \textnormal{Against} Quantum Attacks (LatticeCI)”} (authored by A.~Taximov on 30~May~2025). 
This work proposes a new encryption method called LatticeCI designed to protect sensitive data from future quantum computers. These future computers will be powerful enough to break current security systems. The proposed mathematical algorithm is faster and more efficient than existing standards. This makes it suitable for protecting critical infrastructure like power grids and banking systems from future hacking attempts.
The Organizer is the International Centre for Scientific Cooperation “Science and Education” and proceedings are publicly released with bibliographic data.







\underline{International Scientific Conference \emph{“Modern Technologies Technical and Natural Sciences”} 5 Jun 2025.}

The work includes the Paper \emph{“Optimization of Cloud System Performance Using Adaptive Machine Learning Algorithms”} (authored by A.~Taximov on 5~June~2025). 
The paper looks at ways Machine Learning helps cloud computer systems work faster and stay stable. It sets out methods that forecast when a database will become busy plus that add more computing power on their own before the system fails. It also covers the use of AI to detect data errors without human help and to arrange information in a more efficient way.







\underline{XLIII All Russian Scientific Practical Conference \emph{“Modern Methods and Innovations in Science”} in May~2025.}

The work includes the Paper \emph{“Mechanism of Accelerated Technology Transfer from Scientific Laboratories to Industry of Kazakhstan (TTEP)”} (authored by A.~Taximov in May~2025). 
The proceedings state materials are \emph{selected scientific works} recommended by the Editorial Publishing Council (protocol and date). 
This paper sets out a scheme for a national platform named TTEP that links scientific laboratories with private firms. The aim is to convert research inventions into market ready goods at a faster pace through a digital market for patents. The scheme forecasts a rise of twenty to thirty percent in successful technology launches within three years and gives priority to the launch of green technologies.







\underline{LXXXV International Scientific Practical Conference series \emph{“Scientific Forum Innovative Science”} Issue~6(85), 2025.}



The work includes the Paper \emph{“Methodology of Data Transfer Protocol Standardization for Cross Platform Integration of IoT Devices”} (authored by A.~Taximov in issue 6(85), 2025). 
The proceedings front matter identifies issue and lists the paper while electronic versions are placed in national repositories (such as eLIBRARY or RSCI). 
The paper removes the obstacle that blocks smart devices made by different companies from talking to one another. It puts forward a single standard language and a central hub. Any smart device uses this language plus hub to link up without trouble. The method cuts the technical tangle and trims the time needed to build a smart city or an automated factory by roughly one third.







\underline{International Conference \emph{“Profit Smart City Day 2024”} 11 Oct 2024, Almaty.}



The work includes the Report and Presentation titled \emph{“Smart City The Role of the iKOMEK109 Situation Center in Ensuring City Safety and Identifying Housing and Utility Problems”} (presented by A.~Taximov on 11~October~2024).\\
The Official Invitation Letter along with photographic evidence of the presentation, is a formal proof.
The presentation shows how the city Situation Center works with more than 19000 cameras besides Artificial Intelligence to protect residents. It describes how the system detects emergencies on its own and controls city services like street lighting to cut energy use. By collecting all data in one central platform, the city shortens response times for power cuts or accidents.
Profit Events arranged the gathering, which centered on the digital transformation of Kazakhstan cities and drew government regulators and senior IT managers.











\underline{International Digital Festival \emph{“Google Dev Fest Taraz 2023”} 16-17 Nov 2023, Taraz.}

The work includes the Presentation and Technical Case Study titled \emph{“City Center for Monitoring and Operational Response iKOMEK109 and Project QalaQyzmet (IQALA)”} (presented by A.~Taximov on 16~November~2023). 
The case study explains the creation details of a unified online platform called IQALA. This platform combines all city utility services into one website. Citizens can now handle tasks like property registration or paying bills online through a single window instead of visiting multiple offices. The system eliminates paper documents and uses text messages to keep users updated on their requests. It also ensures that data is transferred securely.
The event was Organized by the Governance of Zhambyl Region in partnership with Google Developers Group and Zhambyl Hub. The goal was to foster regional digitalization and IT talent.





Across these items, Mr.~Taximov’s \emph{own work product} was \textbf{publicly presented} and \textbf{archived} in curated conference proceedings. Since artistic exhibitions do not readily apply to STEM product or program leadership, these conferences are \emph{comparable in significance} under 8~C.F.R.\ §\,204.5(h)(4).











\pagebreak
\subsection{%
  Evidence of \mrl's receipt of lesser nationally or internationally recognized prizes or awards for excellence%
}
\label{subsec:Awards}

% USCIS Criterion 1: Receipt of lesser nationally or internationally recognized prizes or awards for excellence.

% =========================================================
% PRESIDENT — SCO SUMMIT (2024)
% =========================================================
\subsubsection{President of the Republic of Kazakhstan — Letter of Appreciation for the SCO Summit (2024)}
\label{subsubsec:PresSCO2024}

\SubSubSubSection{What was awarded and how it was earned}
In 2024, Mr.~Taximov received a \emph{Letter of Appreciation} signed by the President of the Republic of Kazakhstan for his contribution to the preparation and high-level hosting of the Shanghai Cooperation Organization (SCO) Summit in Astana. The certificate bears the state seal and names Mr.~Taximov personally. \ExhibitRef{PresAlghys2024}

During the summit preparation and live operations, the city’s Monitoring and Rapid Response Center \emph{iKOMEK109}, headed by Mr.~Taximov, ran a 24/7 operational program that combined:
\begin{itemize}
  \item Continuous issue intake from citizens on public safety, utilities, and other city-infrastructure topics (omnichannel contact center).
  \item Citywide digital monitoring workflows (including social-media operators tracking public posts, triaging resident signals, and escalating incidents).
  \item A multi-agency operations hub established on the iKOMEK109 campus to coordinate emergency and law-enforcement services and to synchronize responses to potential threats or incidents in real time.
\end{itemize}
The presidential certificate acknowledges that these processes were organized effectively and sustained under international-event load.

\SubSubSubSection{Why this is in the field (product management)}
The summit operations required end-to-end product ownership of civic platforms: defining workflows, standing up cross-functional incident processes, aligning stakeholders, instrumenting dashboards/alerts, and meeting strict reliability/throughput targets. These are core product-management responsibilities applied to a public-sector service platform.

\SubSubSubSection{Why this is a \emph{lesser nationally or internationally recognized} award}
This is an official, named, state-signed commendation from the head of state tied to an international summit. It is nationally recognized prize.

\ExhibitList{\ExhibitRef{PresAlghys2024}}

\pagebreak

% =========================================================
% PRESIDENT — ELECTION (2022)
% =========================================================
\subsubsection{President of the Republic of Kazakhstan — Letter of Appreciation for the Extraordinary Presidential Election (2022)}
\label{subsubsec:PresElection2022}

\SubSubSubSection{What was awarded and how it was earned}
In November 2022, Mr.~Taximov received an official presidential \emph{Letter of Appreciation} for a significant contribution to the preparation and conduct of the country’s extraordinary presidential election. The document bears the President’s signature and state insignia. \ExhibitRef{PresAlghys2022}

Under Mr.~Taximov’s direction, the capital executed a comprehensive election-technology program:
\begin{itemize}
  \item \textbf{Voter-roll accuracy:} precinct attachment by verified address; joint cleanup of “rubber apartments” with district administrations and the migration service.
  \item \textbf{Geodata correctness:} building geocoding on the digital precinct map within the city’s population database; correction of map layers and coordinates.
  \item \textbf{Register alignment:} assignment of address-register codes, reconciliation with cadastre numbers, and fixes to planning datasets.
  \item \textbf{Final rolls:} production and approval of final voter lists with district administrations and territorial commissions.
  \item \textbf{Voter support at scale:} a unified election call center on iKOMEK109 with a structured knowledge base, trained agents, and scripted escalations.
  \item \textbf{Digital services:} launch of online precinct lookup for voters and analytics for citizen requests connected to election workflows.
  \item \textbf{Data intake pipeline:} validation of lists received from government entities, quasi-public organizations, NGOs, and private companies; processing citizen petitions to add or remove voters.
\end{itemize}
These actions ensured precision and transparency of the rolls and fast, auditable support to residents.

\SubSubSubSection{Why this is in the field (product management)}
The program combined discovery (requirements from commissions and agencies), delivery (data pipelines, services, and support tooling), and operations (SLAs, quality gates, and incident playbooks). Coordinating engineering, data, and service teams around clear success metrics is the essence of product management for civic platforms.

\SubSubSubSection{Why this is a \emph{lesser nationally recognized} award}
A presidential appreciation is a national governmental award linked to a countrywide election. It is broadly recognized in the nation.

\ExhibitList{\ExhibitRef{PresAlghys2022}}

\pagebreak

% =========================================================
% AKIM (MAYOR) — HONORARY DIPLOMA (2022)
% =========================================================
\subsubsection{Mayor  of Astana — ``Honorary Diploma'' for the 2022 Presidential Election}
\label{subsubsec:AkimKurmet2022}

\SubSubSubSection{What was awarded and how it was earned}
The Mayor of Astana awarded Mr.~Taximov an \emph{Honorary Diploma} for \emph{activity, high professionalism, and organizational skill} demonstrated during the preparation and conduct of the election. The certificate is personally addressed, signed, and sealed. \ExhibitRef{AkimKurmet2022}

The recognition covers the city-scale execution led by Mr.~Taximov: verification and attachment of voters to precincts; joint remediation of problematic address records; coordination with territorial commissions; rollout of online precinct-lookup tools and analytics; and establishment of the unified iKOMEK109 election support center.

\SubSubSubSection{Why this is in the field (product management)}
The work centered on owning the roadmap and delivery of mission-critical civic products (voter data services and support channels), managing dependencies across agencies, and measuring outcomes (accuracy, speed, and transparency)—standard product-management practice adapted to elections.

\SubSubSubSection{Why this is a \emph{lesser nationally recognized} award}
Although conferred by the capital’s chief executive, the award honors performance for a \emph{national} electoral event. It is widely recognized within the country and sector.

\ExhibitList{\ExhibitRef{AkimKurmet2022}}

\pagebreak

% =========================================================
% AKIM (MAYOR) — DIGITALIZATION DAY (2022)
% =========================================================
\subsubsection{Mayor of Astana — Letter of Appreciation for the Day of Digitalization and IT Workers (2022)}
\label{subsubsec:AkimAlgys2022}

\SubSubSubSection{What was awarded and how it was earned}
Mr.~Taximov received an official \emph{Letter of Appreciation} from the Akim of Astana for contributions to the development of the city’s digital infrastructure and for introducing modern technologies into city management. \ExhibitRef{AkimAlgys2022}

In 2022, as Head of iKOMEK109, Mr.~Taximov led:
\begin{itemize}
  \item Expansion of citizen channels (mobile app, Telegram bots, and social networks) for 24/7 requests on utilities, maintenance, and public safety.
  \item Strengthening of information-security safeguards across contact-center and online services, protecting personal data and service continuity.
  \item Automation of request processing—moving from a traditional call center to an intelligent, analytics-driven platform—improving speed and transparency.
  \item Delivery of decision-support dashboards and Big-Data tooling for the Mayor’s Office to guide incident response and infrastructure planning.
\end{itemize}

\SubSubSubSection{Why this is in the field (product management)}
The outcomes reflect classic product leadership: multi-channel user experience, backlog and roadmap control, security and reliability requirements, KPIs for responsiveness, and stakeholder alignment with the Mayor’s Office.

\SubSubSubSection{Why this is a \emph{lesser nationally recognized} award}
The Mayor’s letter is a formal governmental commendation in the nation’s capital. It is widely recognized award.

\ExhibitList{\ExhibitRef{AkimAlgys2022}}

\pagebreak

% =========================================================
% KASPERSKY (INTERNATIONAL) — APPRECIATION (2022)
% =========================================================
\subsubsection{Kaspersky Lab (International) — Certificate of Appreciation for Contributions to Information Security (2022)}
\label{subsubsec:KasperskyThanks2022}

\SubSubSubSection{What was awarded and how it was earned}
% =========================================================
% KASPERSKY (INTERNATIONAL) — APPRECIATION (2022)
% =========================================================
\subsubsection{Kaspersky Lab (International) — Certificate of Appreciation for Contributions to Information Security (2022)}
\label{subsubsec:KasperskyThanks2022}

\SubSubSubSection{What was awarded and how it was earned}
Kaspersky Lab’s Central Asia office issued Mr.~Taximov a signed and sealed \emph{Certificate of Appreciation} recognizing his contribution to the development of the information-security sector. \ExhibitRef{KasperskyThanks2022Ex}

The recognition corresponds to Mr.~Taximov’s cybersecurity leadership in practice, including:
\begin{itemize}
  \item Deploying protection and monitoring tools across the iKOMEK109 digital ecosystem that operates citywide, 24/7.
  \item Partnering with leading vendors and integrators to harden critical municipal services and improve resilience.
  \item Establishing incident-response practices and standards in government and quasi-government organizations, including privacy protection.
  \item Promoting digital literacy and secure-by-design approaches for state information systems.
\end{itemize}
This aligns with his concurrent strategic work as an Independent Director at JSC “State Technical Service” (national cyber operator), on projects such as \emph{Synaq} (security testing portal), \emph{ScanKZ} (exposure monitoring), and an internal cyber media-monitoring bot.

\SubSubSubSection{Why this is in the field (product management)}
These activities involve defining problem statements, prioritizing security features, coordinating multi-team delivery, setting acceptance criteria, and measuring impact—product-management applied to cybersecurity platforms and operations.

\SubSubSubSection{Why this is a \emph{lesser internationally recognized} award}
Kaspersky is a globally known cybersecurity company; its signed commendation is professional recognition with international scope. The company’s \textbf{international standing} is independently evidenced by:
\begin{itemize}
  \item \textbf{BBC News coverage of Kaspersky’s research with a UN body.} BBC reports Kaspersky identified the state-level “Flame” malware and conducted research “in conjunction with the UN’s International Telecommunication Union,” calling Flame “one of the most complex threats ever discovered.” \ExhibitRef{KasperskyBBC}
  \item \textbf{Multiple AV-TEST Awards in 2024.} AV-TEST (The Independent IT-Security Institute) states Kaspersky “stood out in numerous tests in 2024, receiving a total of nine awards,” including Windows Best Protection, Best Usability (consumer and corporate), Best Advanced Protection for Corporate Users, and Best macOS Security. \ExhibitRef{KasperskyLabsAwards}
  \item \textbf{SE Labs head-to-head results.} In the \emph{Endpoint Security: Small Business Protection} report (Q1-2024), \emph{Kaspersky Small Office Security} achieved \textbf{100\% Protection Accuracy}, \textbf{100\% Legitimate Accuracy}, and \textbf{100\% Total Accuracy}, demonstrating sustained technical excellence against peer products. \ExhibitRef{KasperskySELabs2024}
\end{itemize}
Together, these independent sources show that recognition from Kaspersky carries \emph{international} weight, satisfying USCIS’s “lesser nationally or internationally recognized” standard for the organization conferring the accolade. The basis for the commendation is \emph{excellence in the field of product-led cybersecurity operations} (see activities above), and the award is \emph{individual} to Mr.~Taximov (not merely his employer).

\ExhibitList{%
  \ExhibitRef{KasperskyThanks2022Ex}, %
  \ExhibitRef{KasperskyBBC}, %
  \ExhibitRef{KasperskyLabsAwards}, %
  \ExhibitRef{KasperskySELabs2024}%
}


\pagebreak

% Exhibits referenced in this section
\ExhibitRef{PresAlghys2024}
\ExhibitRef{PresAlghys2022}
\ExhibitRef{AkimKurmet2022}
\ExhibitRef{AkimAlgys2022}
\ExhibitRef{KasperskyThanks2022Ex}

\pagebreak
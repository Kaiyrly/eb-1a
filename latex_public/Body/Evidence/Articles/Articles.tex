\subsection{Evidence of \mrls authorship of scholarly articles in the field, in professional or major trade publications or other major media}

\label{subsec:Articles}



\smallskip

Mr.~Taximov has authored 13 scholarly and professional articles in smart-city analytics, municipal data engineering, open-data/APIs and civic IT productization.


The "Higher Attestation Commission (VAK)" under the Ministry of Science and Higher Education of the Russian Federation is established for the purpose of ensuring state scientific attestation.
Among its core functions, VAK (a) develops recommendations… on the requirements for peer-reviewed scientific journals and the rules for forming the list of peer-reviewed scientific journals and (b) determines the international databases indexing scientific publications in which the main scientific results of a dissertation may be published. The VAK's journals are included in the list of periodical science and technical publication in Russia recommended by High Attestation Commission for publishing in order to attain Candidate of Science or Doctor of Science degree. \ExhibitRef{VAK_Polozhenie}



Each venue below is represented in the VAK list (as reflected on the journals’ official sites/screenshots attached as exhibits), showing that scholarly works appear in professional outlets with editorial governance and indexing/attestation. See venue-specific quotations and exhibits below.



\underline{Methodology for Creating an Effective IT Ecosystem under Resource Constraints} (A.B.~Taximov; A.A.~Beisenbayev), \emph{Vestnik of Technological University, VAK-listed}, 2025, Vol.~28, No.~4.
Vestnik of Technological University is a VAK-listed scientific journal that publishes research on engineering and information technologies. This article describes a practical way to build a working IT platform and team when budget and time are limited. It explains how to choose the right technologies, how to organize data so the system can handle real workloads, and how to set up a step-by-step launch process where changes are tested and rolled out gradually. It also pays special attention to keeping “unfinished technical problems” under control (technical debt) and choosing a steady working rhythm for the team, so the system stays stable and reliable over time.



\medskip



\underline{Information Technology Tools to Improve Urban Life: iKomek Experience} (A.B.~Taximov; D.R.~Satvaldina), \emph{Components of Scientific and Technological Progress, VAK-listed}, No.~4(94), 2024.
Components of Scientific and Technological Progress is a peer-reviewed scientific journal on science and technology that is published quarterly. This article was published in a peer-reviewed scientific journal that comes out quarterly. It explains how the iKomek center is organized to handle citizen requests. The authors describe how people can contact the center through omnichannel, how appeals are sorted by importance, who is responsible for handling them, and how managers see the results on special dashboards. The article also shows how an appeal moves from the first contact all the way to the city’s main situation center. It points out parts of this system that other cities can copy and use for their own contact centers.



\medskip



\underline{Geonomics: Address-Registry Maintenance and Population Heat-Map Analytics for Smart-City Operations} (A.B.~Taximov; A.A.~Beisenbayev), \emph{Engineering Bulletin of the Don, VAK-listed} (Electronic Scientific Journal), Issue No.~9, 2024.
Engineering Bulletin of the Don is an electronic scientific journal published by the North-Caucasus Research Center of the Southern Federal University. This article introduces the \emph{Geonomics} system, which connects address lists, digital maps, and government databases into one whole system. It helps keep information about where residents live up to date, links each address to the correct polling station during elections, and builds “heat maps” that show where people are concentrated so the city can decide where to place services. The article also explains how the data is cleaned and standardized, how map areas (polygons) are handled, and how all of this is turned into practical reports that help officials make decisions.



\medskip



\underline{Possibilities of Using Open Data: the Concept of e-Democracy in Smart-City Management} (A.B.~Taximov; A.A.~Beisenbayev), \emph{Innovations & Investments, VAK-listed}, No.~11, 2023.
Innovations \& Investments is a VAK-listed journal (category K2) that publishes research on economics, management, and innovation. This article explains how “open data” fits into the way a smart city is managed. It shows how cities moved from simply posting static files online to giving controlled, programmatic access to data through APIs. It connects the way data is published and shared to government transparency, public oversight, and citizen involvement. It also describes rules and practices that make it easy for government bodies, researchers, and developers to safely reuse this data in their own work.



\medskip



\underline{Big Data as a Smart-City Management Tool} (A.B.~Taximov; A.A.~Beisenbayev), \emph{Science and Business: Development Ways, VAK-listed}, Issue 1(151), 2024 (Section: Mathematical Modeling and Numerical Methods).
Science and Business: Development Ways is a peer-reviewed journal used for publishing dissertation results and focuses on science, economics, and business development. This article shows how large amounts of city data (“Big Data”) can be used as a practical management tool. It explains how data analysis methods are connected to tracking key performance indicators (KPIs) and to everyday planning and decision-making. It then describes how the data pipeline is built, how specific indicators are defined, and how the results of the models help decide where to send resources. For example, in transport, social support and other city services. Throughout, it focuses on making the process transparent, repeatable and well-governed.



\medskip



\underline{Information Technologies for Updating and Forming Address Information} (A.B.~Taximov; A.A.~Beisenbayev), \emph{Herald of Computer and Information Technologies (VKIT), VAK-listed}, 2024, pp.~52–62; DOI: \textbf{10.14489/vkit.2024.03.pp.052-062}.
Herald of Computer and Information Technologies (VKIT) is a scientific, technical, and production monthly journal focused on computer and information technologies. It describes a program for engineering the election-period data. The program increased voter-list accuracy from 84\% to 100\% through address-code verification, geocoding, normalization, and deduplication. The article sets out the procedures, control points, and portability considerations for similar projects.



\medskip



\underline{Strategy to Improve the International Transfer of Cybersecurity Technologies} (A.B.~Taximov), \emph{Automation. Modern Technologies, VAK-listed}, 2023, Vol.~80, No.~\dots, pp.~\dots–\dots; DOI: \textbf{10.36652/0869-4931-2026-80-\dots-\dots-\dots}.
Automation. Modern Technologies is a long-running peer-reviewed scientific and technical journal on automation and modern information technologies. This article is published in a long-running scientific journal that focuses on automation and modern IT, and is reviewed by other experts before publication. The study explains how countries can share and adapt cybersecurity technologies when there are sanctions and limited access to foreign products. It suggests a practical way for scientists and businesses to work together, using engineering centers and open-source tools. The article also lists the main obstacles to this kind of cooperation and offers step-by-step recommendations to make national cybersecurity stronger.



\medskip



\underline{Faster Decision-Making withtechnology to Integrate Artificial Intelligence into Smart-City Control Centers} (A.B.~Taximov), \emph{Information Systems and Technologies, VAK-listed}, 2024, No.~\dots, pp.~\dots–\dots.
Information Systems and Technologies is a peer-reviewed scientific journal of Orel State University that focuses on information systems and information technologies. This article is published in a peer-reviewed scientific journal of Orel State University that focuses on information systems and technologies. The work addresses a problem in contact centers. In instance, staff in smart-city control rooms receive too much information and can struggle to react quickly and accurately. The article suggests a new model that uses artificial intelligence to assist them. In this model, the system automatically collects data, predicts what might happen, analyzes camera images, and uses a built-in knowledge base of typical situations and responses. Operators see all this through a smart interface that helps them choose actions faster. The goal is to shorten reaction times, reduce mental overload for staff, and improve the quality of decisions in the control center.


\pagebreak